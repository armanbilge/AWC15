\documentclass{beamer}
\usepackage{graphicx}

\definecolor{uoanavy}{RGB}{0,60,127}

\usetheme{Frankfurt}
\usecolortheme{orchid}
\setbeamertemplate{navigation symbols}{}
\setbeamercolor{structure}{fg=uoanavy}

\usepackage{pgfplots}
\usepackage{pgfplotstable}
\pgfplotsset{compat=newest,
             jitter/.style={
                 x filter/.code={\pgfmathparse{\pgfmathresult+rnd*#1-#1/2}}
             },
             jitter/.default=0,
             select coords between index/.style 2 args={
                 x filter/.code={
                     \ifnum\coordindex<#1\def\pgfmathresult{}\fi
                     \ifnum\coordindex>#2\def\pgfmathresult{}\fi
                 }
             }
 }

\usepackage{doi}

% \usepackage{enumitem}
% \setlist[itemize]{label=$\triangleright$}

\title{Implementing Hamiltonian Monte Carlo for Efficient Bayesian Evolutionary Analysis}
\author{Arman Bilge}

\institute{Computational Evolution Group \\ The University of Auckland}
\date{21 October 2015}

\usepackage{mathtools}
\newcommand{\dd}{\, \mathrm{d}}
\renewcommand{\vec}[1]{\ensuremath{\boldsymbol{\mathbf{#1}}}}
\newcommand{\mat}[1]{\ensuremath{\boldsymbol{\mathbf{#1}}}}
\newcommand{\op}[1]{\ensuremath{\boldsymbol{\mathbf{#1}}}}
\newcommand{\norm}[1]{\ensuremath{\mathcal{N}\left(#1\right)}}

\begin{document}

    \frame{\titlepage}

    \section{Introduction}

    \stepcounter{subsection}
    \begin{frame}{Motivation}

        \begin{itemize}

            \item Bayesian statistics has transformed evolutionary biology
            \begin{itemize}
                \item Divergence dating
            \end{itemize}

        \end{itemize}

    \end{frame}

    \stepcounter{subsection}
    \begin{frame}{Bayesian Evolutionary Analysis}
    \end{frame}

    \stepcounter{subsection}
    \begin{frame}{Hamiltonian Dynamics}
        \begin{columns}
            \begin{column}{.5\textwidth}
                \begin{definition}
                    Let
                    \small
                    \begin{itemize}
                        \item $\vec{q}$ be the skier's position
                        \item $\vec{v}$ be their velocity
                        \item $\mat{M}$ be their mass
                        \item $U\left(\vec{q}\right)$ be their potential energy
                        \item $K\left(\vec{p}\right)$ be their kinetic energy
                    \end{itemize}

                    \normalsize
                    Then the Hamiltonian $\mathcal{H}$ is
                    \begin{equation*}
                        \mathcal{H}\left(\vec{q},\vec{p}\right) = U\left(\vec{q}\right) + K\left(\vec{p}\right)
                    \end{equation*}
                \end{definition}
            \end{column}
            \begin{column}{.5\textwidth}

            \end{column}
        \end{columns}
    \end{frame}

    \section{Methods}

    \stepcounter{subsection}
    \begin{frame}{Hamiltonian Monte Carlo}
    \end{frame}

    \stepcounter{subsection}
    \begin{frame}{Hamiltonian Monte Carlo}
    \end{frame}

    \stepcounter{subsection}
    \begin{frame}{Hamiltonian Monte Carlo}

        \textbf{Running a physics simulator seems like a lot of work! \\ Why should we bother?}

        \begin{theorem}[Creutz 1988]
            Consider a model with $n$ variables. \\
            Then (under simplifying assumptions) the computation time is
            \begin{itemize}
                \item $\mathcal{O}\left(n^2\right)$ for MCMC; and
                \item $\mathcal{O}\left(n^\frac{5}{4}\right)$ for HMC.
            \end{itemize}
        \end{theorem}

        In practice this means \textbf{doubling the model complexity} increases computation time by
        \begin{itemize}
            \item \textbf{4x} for MCMC
            \item \textbf{$<$2.5x} for HMC!
        \end{itemize}

    \end{frame}

    \section{Results}

    \stepcounter{subsection}
    \begin{frame}{A Phylogenetic Hamiltonian}
    \end{frame}

    \stepcounter{subsection}
    \begin{frame}{Performance of HMC vs. MCMC}

        For $\left\{8,16,32,64\right\}$ taxa:

        \begin{itemize}
            \item Simulated 100 datasets under Yule and HKY models
            \item Estimated node heights with optimally-tuned HMC and MCMC
            \item Measured efficiency as effective sample size of tree length per unit time
            \item ESS = number of independent samples
        \end{itemize}

    \end{frame}

    \stepcounter{subsection}
    \begin{frame}{Performance of HMC vs. MCMC}
        \centering
        \begin{tikzpicture}
            \begin{semilogxaxis}[width=0.9\textwidth,
                                 xtick={8,16,32,64},
                                 ytick={1,4,8,...,20},
                                 extra y ticks=1,
                                 extra y tick labels=,
                                 extra y tick style={ grid=major, major grid style={thick} },
                                 log ticks with fixed point,
                                 xlabel=Number of Taxa,
                                 ylabel=Relative Efficiency of HMC,
                                 clip mode=individual
                         ]
                \addplot+[uoanavy, only marks, mark size=1, mark options={fill=uoanavy}, jitter=0.25] table {performance.dat};
                \draw[red, thick] (7, 4.21355) -- (9.14285714285714, 4.21355);
                \draw[red, thick] (14, 4.776769) -- (18.28571, 4.776769);
                \draw[red, thick] (28, 5.342736) -- (36.57143, 5.342736);
                \draw[red, thick] (56, 5.547638) -- (73.14286, 5.547638);
            \end{semilogxaxis}
        \end{tikzpicture}
    \end{frame}

    \section{Conclusion}

    \stepcounter{subsection}
    \begin{frame}{Acknowledgements}
        \begin{block}{Thanks to}
            \begin{itemize}
                \item Tim Vaughan and Alexei Drummond
                \item Members of the Computational Evolution Group
                \item Allan Wilson Centre Summer Scholarship
                \item New Zealand eScience Infrastructure
            \end{itemize}
        \end{block}
        \begin{block}{References}
            \footnotesize
            M Creutz. \textit{Physical Review D} 38.4 (1988). \texttt{\doi{10.1103/PhysRevD.38.1228}} \\
            RM Neal. \textit{Handbook of Markov Chain Monte Carlo} (2011).
        \end{block}
    \end{frame}

\end{document}
